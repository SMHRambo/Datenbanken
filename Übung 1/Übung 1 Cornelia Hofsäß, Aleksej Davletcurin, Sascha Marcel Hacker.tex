\documentclass[ngerman]{gdb-aufgabenblatt}


\renewcommand{\Aufgabenblatt}{1}
\renewcommand{\Ausgabedatum}{Mi. 15.10.2014}
\renewcommand{\Abgabedatum}{Do. 31.10.2014}
\renewcommand{\Gruppe}{Cornelia Hofs�\ss{}, Aleksej Davletcurin, Sascha Marcel Hacker}
\renewcommand{\STiNEGruppe}{30}


\begin{document}
\section{Informationssysteme}
\subsection{a) Charakterisierung:}
Ein rechnergest�tztes Informationssystem ist ein System, 
bei dem die Erfassung, Speicherung und/oder Transformation von Informationen durch den Einsatz von EDV teilweise automatisiert ist.1 Die Aufgaben sind Erfassung, Speicherung, Transformation von Informationen.

\subsection{b) Datenunabh�ngigkeit:}
Die Datenunabh�ngigkeit beschreibt die Unabh�ngigkeit der Daten im 3-Schichtenmodell(physikalische Schicht(interne Schicht), konzeptionelle Schicht, externe Schicht(Benutzersichten)).

Physische Datenunabh�ngigkeit bedeutet, dass die interne von der konzeptionellen und   externen Ebene getrennt ist. Physische �nderungen, z.B. des Speichermediums oder des Datenbankprodukts, wirken sich nicht auf die konzeptionelle oder externe Ebene aus.

Logische Datenunabh�ngigkeit hei�t, dass die konzeptionelle und die externe Ebene getrennt sind. Dies bedeutet, dass �nderungen an der Datenbankstruktur (konzeptionelle Ebene) keine Auswirkungen auf die externe Ebene, also die Masken-Layouts, Listen und Schnittstellen haben.

\subsection{c)}


\section{Miniwelt}
\subsection{a)}
Objekttypen
	Mitspieler
	Tippgemeinschaft
	Wettbewerb
	Begegnung
	(Ergebnis)
	(Punkte)
Vorg�nge
	Mitglieder erstellen
	Mitglieder l�schen
	Tippgemeinschaft erstellen
	Tippgemeinschaft l�schen
	Wettbewerb erstellen (wird gel�scht wenn Tippgemeinschaft gel�scht wird)
	Begegnung erstellen (wird gel�scht wenn Wettbewerb gel�scht wird)
	Tipp abgeben
	Mitglieder zu Tippgemeinschaft hinzuf�gen
	Mitglieder aus Tippgemeinschaft entfernen
	Ergebnis eintragen
	Punkte berechnen
	Punkt ausgeben
	
\subsection{b)}


\section{Transaktionen}
\subsection{1. Fall:}
	Stromausfall A\\
	Konto 5: Unver�ndert\\
	Konto 7: Unver�ndert\\
	Der Befehl wurde nicht gespeichert
\subsection{2. Fall:}
	Stromausfall A\\
	Konto 5: -1000\\
	Konto 7: Unver�ndert\\
	Der Befehl wurde gespeichert
\subsection{3. Fall:}
	Stromausfall B\\
	Konto 5: Unver�ndert\\
	Konto 7: Unver�ndert\\
	Beide Befehle wurden nicht gespeichert
\subsection{4. Fall:}
	Stromausfall B\\
	Konto 5: -1000\\
	Konto 7: Unver�ndert\\
	Nur der erste Befehl wurden gespeichert
\subsection{5. Fall:}
	Stromausfall B\\
	Konto 5: -1000\\
	Konto 7: +1000\\
	Beide Befehle wurden gespeichert
\subsection{6. Fall:}
\subsection{7. Fall:}
\subsection{8. Fall:}

Softwareseitig: Indem ein Journal und Dirtyflags eingesetzt werden.
Hardwareseitig: Indem Speichercontroller mit Batterycache oder USV(unterbrechungsfreie Stromversorgung) eingesetzt werden.

\section{Warm-Up MySQL}
\subsection{a)}
1. Es wird eine Tabelle mit dem Namen gdb\_gruppeG30.user erstellt die 3 Felder hat.
\begin{itemize}
\item id vom Typ Integer und Primary Key.
\item name vom Typ VarChar mit einer maximalen L�nge von 49 Zeichen der nicht NULL sein darf.
\item password vom Typ VarChar mit einer maximalen L�nge von 8 Zeichen der nicht NULL sein darf.
\end{itemize}
	
2. Es wird ein Datensatz/Tupel in die Tabelle gdb\_gruppeG30.user eingef�gt/hinzugef�gt mit folgenden Werten:
\begin{itemize}
\item 1 als id.
\item gdbNutzer als name.
\item geheim als password.
\end{itemize}

\subsection{b)}
1. Es werden alle Datens�tze/Tupel der Tabelle gdb\_gruppeG30.user aufgegeben bei denen der das Attribut Name den Wert gdbNutzer hat.

2. Die Tabelle gdb\_gruppeG30.user wird von Datenbank gel�scht.
\subsection{c)}

\end{document}

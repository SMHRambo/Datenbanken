\documentclass[ngerman]{gdb-aufgabenblatt}


\renewcommand{\Aufgabenblatt}{4}
\renewcommand{\Ausgabedatum}{Mi. 15.10.2014}
\renewcommand{\Abgabedatum}{Do. 31.10.2014}
\renewcommand{\Gruppe}{Cornelia Hofs�\ss{}, Aleksej Davletcurin, Sascha Marcel Hacker}
\renewcommand{\STiNEGruppe}{30}



\begin{document}

$\pi_{Jahresgehalt}$(Job $\underset{JNR=Job}{\bowtie}$ Bewerbung $\underset{Bewerber=PNR}\bowtie$ ($\sigma_{Geb}$ $\geq$ '1980-01-01'(Person)))
\\

$\pi_{Titel,Jahresgehalt}$(Job $\underset{JNR=Job}{\bowtie}$Bewerbung$\underset{Bewerber=PNR}\bowtie$Person$\underset{Heimat=LNR}\bowtie$($\sigma_{Name='Schweiz'}(Land)$))
\\

$\pi_{Vorname,Nachname}$(Personen$\bowtie(\pi_{PNR}(Personen)-\pi_{PNR}(Bewerber)) )$
\\

Gebe das Geburtsdatum der Personen aus, deren Sachbearbeiter nach dem 31.12.1994 geboren worden sind.
\\

Dadurch das die referentielle Intergrit�t von Fremdschl�sseln nicht verz�gert am Ende der Transaktion gepr�ft werden kann, k�nnen keine wechselseitige Abh�ngigkeiten aufgebaut werden.
Wenn Buch noch das Atribut Editor enthalten w�rde, das auf Person.PID zeigt, dann k�nnten beide Tabellen nicht erstellt werden, da sie von der jeweils anderen abh�ngen.
Eine L�sung w�re den Fremdschl�ssel sp�ter einzuf�gen mit dem Befehl alter table.




\end{document}
